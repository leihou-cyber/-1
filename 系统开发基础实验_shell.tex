\documentclass[a4paper, 12pt]{article}
\usepackage[UTF8]{ctex}
\usepackage{color}
\usepackage{graphicx}

\begin{document}
	\title{\color{red}{\huge 实验报告}}
	\author{\color{blue}{\large 周雍凡}}
	\date{\today}
	\maketitle
	
	\pagenumbering{roman}
	\tableofcontents
	\newpage
	
	\pagenumbering{arabic}
	\setcounter{page}{1}
	
	\section{pwd}
	{\color{blue} 命令格式:pwd}
	\paragraph{获取当前工作目录}
        
	\section{cd}
	{\color{blue} 命令格式:cd /home}
	\paragraph{切换目录}
    \begin{figure}[h]
        \centering
        \includegraphics[width=1.0\linewidth]{Snipaste_2024-08-31_16-00-31.png}
        \caption{cd 示例}
        \label{fig:cd示例}
    \end{figure}
	
	\section{mkdir}
	{\color{blue} 命令格式:mkdir missing}
	\paragraph{创建文件夹}
    \begin{figure}[h]
        \centering
        \includegraphics[width=1.0\linewidth]{Snipaste_2024-08-31_15-55-31.png}
        \caption{mkdir 示例}
        \label{fig:mkdir示例}
    \end{figure}
	
	\section{touch}
	{\color{blue} 命令格式:touch test.txt}
	\paragraph{创建文件}
    \begin{figure}[h]
        \centering
        \includegraphics[width=1.0\linewidth]{Snipaste_2024-08-31_16-00-45.png}
        \caption{touch 示例}
        \label{fig:touch示例}
    \end{figure}

	\section{--help/man}
	{\color{blue} 命令格式:touch --help/man touch}
	\paragraph{获取帮助手册}
	
	\section{echo >}
	{\color{blue} 命令格式:echo hello > hello.txt}
	\paragraph{> 输入流,将信息输入文件}
    \begin{figure}[h]
        \centering
        \includegraphics[width=1.0\linewidth]{Snipaste_2024-08-31_16-04-47.png}
        \caption{> 示例}
        \label{fig:>示例}
    \end{figure}
	
	\section{cat <}
	{\color{blue} 命令格式:cat < hello.txt}
    \paragraph{< 输出流,通过 cat 将文件中信息输出}	
    \begin{figure}[h]
        \centering
        \includegraphics[width=1.0\linewidth]{Snipaste_2024-08-31_16-06-18.png}
        \caption{< 示例}
        \label{fig:<示例}
    \end{figure}

	\section{>>}
	{\color{blue} 命令格式:echo hello >> hello.txt}
	\paragraph{追加内容}
    \begin{figure}[h]
        \centering
        \includegraphics[width=1.0\linewidth]{Snipaste_2024-08-31_16-05-00.png}
        \caption{>> 示例}
        \label{fig:>>示例}
    \end{figure}
	
	\section{| }
	{\color{blue} 命令格式:ls -l / | tail -n1}
	\paragraph{管道,更好的利用文件重定向。| 操作符允许我们将一个程序的输出和另外一个程序的输入连接起来}
		
	\section{“ ”}
	{\color{blue} 命令格式:echo "test"}
	\paragraph{引用,双引号包围的内容可以允许变量扩展,也允许转义字符的存在}
	
	\section{’‘}
	{\color{blue} 命令格式:echo ''}
	\paragraph{将引号内内容原封不动显示}
    \begin{figure}[h]
        \centering
        \includegraphics[width=1.0\linewidth]{Snipaste_2024-08-31_16-09-51.png}
        \caption{'' 示例}
        \label{fig:''示例}
    \end{figure}

	\section{ls}
	{\color{blue} 命令格式:ls -l}
	\paragraph{命令用于显示指定工作目录下之内容(列出目前工作目录所含的文件及子目录)}
    \begin{figure}[h]
        \centering
        \includegraphics[width=1.0\linewidth]{Snipaste_2024-08-31_16-12-36.png}
        \caption{ls 示例}
        \label{fig:ls示例}
    \end{figure}

    \section{使用 | 和 > ,将 semester 文件输出的最后更改日期信息,写入文件中}
    \paragraph{./semester | grep last-modified > ~/last-modified.txt}

    \section{Powercfg -energy}
    \paragraph{生成电池的使用情况报告}

    \section{ls -a\\
    ls -h\\
    ls -t\\
    ls --color=auto}
    \paragraph{-a: 所有文件(包括隐藏文件)\\
    -h: 文件打印以人类可以理解的格式输出\\
    -t: 文件以最近访问顺序排序\\
    --color=auto: 以彩色文本显示输出结果}
    \newpage
    
    \section{marco,polo 执行操作}
    \paragraph{vim marco.sh}
    \paragraph{用 vim 编辑器创建 .sh 文件用于初始化函数}
    \paragraph{source marco.sh}
    \paragraph{加载 marco.sh 文件}
    \begin{figure}[h]
        \centering
        \includegraphics[width=1.0\linewidth]{Snipaste_2024-08-31_16-43-55.png}
        \caption{创建到加载}
        \label{fig:创建到加载}
    \end{figure}

    \paragraph{编写两个函数}
    \begin{verbatim}
marco(){
     echo "$(pwd)" > $HOME/marco_history.log
     echo "save pwd $(pwd)"
 }
 polo(){
     cd "$(cat "$HOME/marco_history.log")"
 }
    \end{verbatim}

    \paragraph{执行函数 marco 与 polo}
    \begin{figure}[h]
        \centering
        \includegraphics[width=1.0\linewidth]{Snipaste_2024-08-31_16-44-15.png}
        \caption{执行 marco}
        \label{fig:执行marco}
    \end{figure}
    \begin{figure}[h]
        \centering
        \includegraphics[width=1.0\linewidth]{Snipaste_2024-08-31_16-44-34.png}
        \caption{执行 polo}
        \label{fig:执行polo}
    \end{figure}
    \paragraph{可以看到执行完 polo 后所在文件夹位置发生了变化}
    \newpage

    \section{函数的debug}
    \paragraph{test.sh和debug_for.sh函数}
    \begin{figure}[h]
        \centering
        \includegraphics[width=0.75\linewidth]{Snipaste_2024-08-31_16-54-20.png}
        \caption{test.sh 函数}
        \label{fig:test.sh}
    \end{figure}
    \begin{figure}
        \centering
        \includegraphics[width=1\linewidth]{Snipaste_2024-08-31_16-56-09.png}
        \caption{debug_for.sh函数}
        \label{fig:enter-label}
    \end{figure}
    
    
    
    \paragraph{循环直到出现错误}
    \begin{figure}[h]
        \centering
        \includegraphics[width=0.75\linewidth]{Snipaste_2024-08-31_16-57-43.png}
        \caption{执行过程}
        \label{fig:执行过程}
    \end{figure}

    \section{find 指令}
    \paragraph{首先创建文件以及文件夹}
    \begin{figure}[h]
        \centering
        \includegraphics[width=0.75\linewidth]{Snipaste_2024-08-31_16-59-08.png}
        \caption{创建文件目录}
        \label{fig:创建文件目录}
    \end{figure}
    \paragraph{使用 find 指令将当前文件夹及其子文件夹中带有 .html 后缀的文件压缩至名为 html 的 zip 压缩包中}
    \begin{figure}[h]
        \centering
        \includegraphics[width=1.0\linewidth]{Snipaste_2024-08-31_16-59-32.png}
        \caption{执行过程}
        \label{fig:执行过程}
    \end{figure}
    
\end{document}
