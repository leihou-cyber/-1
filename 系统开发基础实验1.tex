\documentclass[a4paper, 12pt]{article}
\usepackage[UTF8]{ctex}
\usepackage{color}
\begin{document}
	\title{\color{red}{\huge 实验报告}}
	\author{\color{blue}{\large 周雍凡}}
	\date{\today}
	\maketitle
	
	\pagenumbering{roman}
	\tableofcontents
	\newpage
	
	
	\section{git init}
	
	{\color{blue} 命令格式:git init}
	\paragraph{初始化一个Git仓库,执行完git init命令后,会生成一个.git目录,该目录包含了资源数据,且只会在仓库的根目录生成。}
	\section{git clone}

	{\color{blue} 命令格式:git clone <url> [directory]}
	\paragraph{clone命令可以从Git仓库拷贝项目,
	如git clone git://github.com/schacon/grit.git}
	
	\section{git add}
	{\color{blue} 命令格式:git add}
	将文件添加到缓存,如新项目中
	
	\section{git commit}
	{\color{blue} 命令格式:git commit -m "第一次版本提交"}
	将缓存区内容添加到仓库中,可以在后面加-m选项,以在命令行中提供提交注释


	\section{git mv}
	{\color{blue} 命令格式:git mv test.txt newtest.txt}
	用于移动或重命名一个文件、目录、软连接,如要将一个test.txt文件重命名为newtest.txt
	
	\section{git branch}
	{\color{blue} 命令格式:git branch/git branch branchname}
	可以查看分支,也可以创建分支,如果没有参数时,git branch会列出你在本地的分支;如果有参数时,git branch就会创建改参数的分支
	
	\section{git checkout}
	{\color{blue} 命令格式:git checkout branchname}
	切换分支,git checkout test
	
	\section{git merge}
	{\color{blue} 命令格式:git merge branchname}
	将任意分支合并到到当前分支中去,git merge test
	
	\section{git branch}
	{\color{blue} 命令格式:git branch -d (branchname)}
	删除分支,git branch -d test
		
	\section{git remote add}
	{\color{blue} 命令格式:git remote add [alias] [url]}
	添加一个远程仓库,参数[alias]为别名, [url]为远程仓库的地址
	git remote add test https://github.com/test.git
			
	\section{git pull}
	{\color{blue} 命令格式:git pull [options] [<repository> [<refspec>…]]}
	相当于是从远程获取最新版本并merge到本地	
	如:git pull test master --allow-unrelated-histories(本地与远程独立时)
			
	\section{git push}
	{\color{blue} 命令格式:git push [alias] [branch]}
	推送你的新分支与数据到某个远端仓库命令
	如:git push test master
	
\end{document}